\documentclass[document.tex]{subfiles}
\begin{document}

\chapter{2023-07-17 (Weekday)}
\label{day:2023-07-17}
\section*{Daily Activities}
\subsection*{Goals}


\section{A1:Genomes & Biodiversity (Evolutionary Genetics)}
    
        \subsection{The role of structural genomic variants in the evolution of biodiversity, Claire Mérot, CNRS ECOBIO / Université de Rennes, France}
        \begin{itemize}
        \item antagonistic pliotrophy - life hist trade surv-repo - seweed fly is polymorphic
        \item  low cov - gen var - over env 
        \item high LD indicating lack of recom (age of div indicated by overall divergence) 
        \item sliding window analysis
        \item lake White fish -2 sp
        \item - test ch lev sv with 30x
        \item assosc with TE, Rattus vs Rattus
        \item SV can be a genome wide syndrome- accumulative or resp for indirect impacts- TAD in mice!
        \item  
        \end{itemize}

        \subsection{Drivers of diversification in a neotropical biodiversity hotspot: lessons from spiders, birds, and butterflies, Carolina Pardo-Diaz, Universidad del Rosario, Colombia}
        \begin{itemize}
        \item delimitR package
        \item 
        \item 
        \end{itemize}

        \subsection{Pangenomes of North American Scrub-Jays (Aphelocoma) reveal abundant structural variation and rapid shifts in genome size, Scott Edwards, Harvard University, USA}
        \begin{itemize}
        \item check photo for study/paper
        \item tools for building pan-genomes: pggb, odgi 
        \item "large" SVs deleterious in the jay 
        \end{itemize}

        \subsection{Genomic basis of body miniaturization in Southern pygmy perch (Nannoperca australis; Teleostei), Jonathan Sandoval Castillo, Flinders University, Australia}
        \begin{itemize}
        \item 
        \item 
        \item 
        \end{itemize}


        \subsection{Integrating genomics, conservation, and Indigenous knowledge to protect a unique marsupial, Carolyn Hogg, University of Sydney, Australia}
        \begin{itemize}
        \item CAFE analysis 
        \item 
        \item 
        \end{itemize}

        \subsection{Biodiversity genomics of asexual and anhydrobiotic nematodes in extreme environments, Philipp Schiffer, University of Cologne, Germany, Biodiversity genomics of asexual and anhydrobiotic nematodes in extreme environments}
        \begin{itemize}
            \item 
            \item 
            \item 
            \end{itemize} 





    \section{B2: Population Genetics & Genomics (Genetics & the Environment)}

        \subsection{Going beyond SNPs: structural variants as facilitators of eco-evolutionary change, Maren Wellenreuther, Plant & Food Research, New Zealand}
            \begin{itemize}
                \item SW fly, how big is the inversion, how was it detected?
                \item Snapper LM, Opticla mapping? + G assembly
                \item Lumpy + GATK on 250bp HiSeq4000
                \item SV 3x bp impact + impacts 3% of all genes 
                \item long read for rats - pangenome with 5-10 samp
                \item most SV > 500bp
                \item high genome quality needed for larger SV detection
                \item what env factors?
                \end{itemize} 

        \subsection{Adaptation to natural and urban environments: a transposable element perspective,Josefa Gonzalez, Institute of Evolutionary Biology, Spain }
            \begin{itemize}
                \item 
                \item 
                \item 
                \end{itemize} 

        \subsection{Genetic and phenotypic consequences of local losses of sexual reproduction in the wild, oleille Miller, University of New South Wales, Australia
        Recipient of the GSA Smith-White Travel Award}
            \begin{itemize}
                \item 
                \item 
                \item 
                \end{itemize} 

        \subsection{The genomic impact of island isolation in Australian mammals, Emily Roycroft, Australian National University, Australia}
            \begin{itemize}
                \item 
                \item 
                \item 
                \end{itemize} 

        \subsection{The maintenance of alternative fitness peaks in the face of gene flow, David Field, Macquarie University, Australia}
            \begin{itemize}
                \item 
                \item 
                \item 
                \end{itemize} 

        \subsection{Lightning talks}
            \begin{itemize}
                \item 
                \item 
                \item 
                \end{itemize} 


                %notes to self: collab with australian mouse virology group, also stephans student-zebrafish gene drive poster

                Heng Lin Yeap CSIRO Australia FEASIBILITY OF CRISPR-CAS GENE DRIVE FOR POPULATION CONTROL: A CASE STUDY IN ORYCTOLAGUS CUNICULUS IN AUSTRALIA 079

                Yiran Liu Peking University China ADVERSARIAL INTERSPECIES RELATIONSHIPS FACILITATE POPULATION SUPPRESSION BY GENE DRIVE IN SPATIALLY EXPLICIT MODELS 046

                Benjamin Camm University of Melbourne Australia MODELLING GENE DRIVE LOCALISATION THROUGH GENETIC VARIATION 007

                Gelshan I Godahewa University of Adelaide Australia DEVELOPMENT OF AN X SHREDDER GENE DRIVE FOR SUPPRESSION OF INVASIVE MICE 027

                Wei-Shan Vivi Chang CSIRO Australia EXTENSIVE GENOMIC DIVERSITY OF RODENT-BORNE CORONAVIRUSES HARBOURED BY AUSTRALIAN WILD MICE

                Clancy Lawler University of Melbourne Australia: STEPS TOWARDS THE GENERATION OF A FUNCTIONAL GENE DRIVE IN ZEBRAFISH 044


   
    \bib{}
\end{document}
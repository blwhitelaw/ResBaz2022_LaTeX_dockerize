\documentclass[twocolumn, letterpaper]{scrartcl}

\usepackage{uog_factsheet}

\newcommand\sourceURL{https://www.overleaf.com/read/xmyythnqhymz}
\newcommand\docID{2018-00}
\newcommand\createdOn{April 16, 2018}

\begin{document}
    \title{\color{triton_green}Ship Rat,\\ \textit{Rattus rattus} (Rodentia: Muridae)}
    \author{Brooke L Whitelaw}
    \date{}

	\maketitle
	
	
	\begin{figure}[tbp]
		\includegraphics[width=\linewidth]{example-image-a}
		\caption{The letter A in a box. \label{fig:a}}
		\vspace{0.1in}
		\includegraphics[width=\linewidth]{example-image-b}
		\caption{flowchart 2. \label{fig:b}}
	\end{figure}	
    
    \section*{Native range and expansion}

    	The ship rat also known as the black rat originates from X
        
    \section*{Invasion of Aotearoa}
        The first inavasion has been dated to X ship 
	
    
    \section*{Evironmental impact}
        The isolation of Aotearoa from mammalian pedators has left many native species vulnerale to direct or indirect impact from these small oppertunistic animals.

    \section*{Science}
        What are we going to do about it?
        Current control methods invlove the use of a toxin called X which is an anticoagulent fatal to rodents. However, this approch does not resitict its impact to the invasive rodents ... has colatoral damage 
        Alternatives? 
        intoducing infertility into the population 

% https://en.wikibooks.org/wiki/LaTeX/Bibliography_Management#Embedded_system
    \bibliographystyle{unsrtnat}
    %\bibliography{biblio}
    
    \begin{thebibliography}{9}
        

    \end{thebibliography}
    
	\blurb
\end{document}
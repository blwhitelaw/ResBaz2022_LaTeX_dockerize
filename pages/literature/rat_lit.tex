\documentclass[document.tex]{subfiles}
\begin{document}


\section{Invasion Genetics}
    \subsection{\citetitle{carneiro2020rapid}}
    \begin{itemize}
    \item Role of epigenetic adaptation in invasive species 
    \item "A particularly interesting example is provided by the marbled crayfish, a novel, monoclonal freshwater crayfish species that has colonized diverse habitats within a few years"
    \item "When compared to the parent species of marbled crayfish (P. fallax), many genes showed reduced levels of gene body methylation and increased levels of gene expression variability (Gatzmann et al. 2018)."
    \end{itemize}


    \subsection{\citetitle{white_2012}}
    \begin{itemize}
    \item Invasive species may hav a decresed parasite load as they have less exposure to both parasitic and pathonogentic organisms in their native range.
    \item Low pathogen load attribiuted to low host density and founder effects
    \item Resources can be reallocated from disease protection to other processes which may facillitate greater survival and invasion.
    \item 
    \end{itemize}

    \subsection{\citetitle{keane_2002}}
    \begin{itemize}
    \item Opinion article on the Enemy Release Hypothesis for invasive plants 
    \item "One commonly accepted mechanism for exotic plant invasions is the enemy release hypothesis (ERH), which states that plant species, on introduction to an exotic region, experience a decrease in regulation by herbivores and other natural enemies, resulting in a rapid increase in distribution and abundance"
    \end{itemize}


    \subsection{\citetitle{}}
    \begin{itemize}
    \item 
    \item 
    \end{itemize}

\section{pylogenetics and distribution of invasive rats in Aotearoa}

    \subsection{\citetitle{wilson2007estimating}}
    \begin{itemize}
    \item 
    \item The presence of ship rats significantly increased with the presence of a Wharf or if an island was Inhabited
    \item Norway rat presence was negatively correlated with relative seabird species richness, and positively correlated with relative exotic land bird species richness
    \item As for Norway rats, there appears to be a negative interaction between rodent species that affects the distribution of ship rats. However, in this case the relationship is not as significant as it was for Norway rats. This may be because ship rats are the superior competitor in New Zealand (Atkinson, 1986; Yom-Tov et al., 1999; Innes, 2001).
    \end{itemize}

    \subsection{\citetitle{abdelkrim2010fine}}
    \begin{itemize}
    \item Fine scale gentic structure of invasive black rats on mainland NZ using microstatts
    \item Little genetic structure over 5km2, however a weak but significant IBD was detected
    \item Isolation was ot found between sample sites and external sites (>20km)
    \item 
    \end{itemize}

    \subsection{\citetitle{king_2011}}
    \begin{itemize}
    \item "Genotyping of 493 carcases found no significant population-level differentiation among the 8 fragments, confirming that the rats in all fragments belonged to a single dynamic metapopulation"
    \item "Marked rats of both genders travelled up to 600 m in a few days"
    \item (1) the replacement rats were derived from a meta-population so large that our samples, spread over 20,000 km2, could not define its boundaries; (2) gene flow between forest fragments within the meta-population was not inhibited by large areas of intervening non-preferred habitat, so a few hundred metres of pasture clearly could not protect our study areas from reinvasion after eradication; and (3) movements of individual rats towards the cleared fragments from outside were not stimulated by human interference–on the contrary, we detected substantial numbers of marked rats on the very first day of trapping, suggesting an ongoing dynamic interaction between the rats living in the fragments and in neighbouring areas that was already in progress before our first eradications began to take effect.
    \end{itemize}

    \subsection{\citetitle{russell_2010}}
    \begin{itemize}
    \item 
    \item 
    \end{itemize}

    \subsection{\citetitle{yarita_2023}}
    \begin{itemize}
    \item 
    \item 
    \end{itemize}

    \subsection{\citetitle{}}
    \begin{itemize}
    \item 
    \item 
    \end{itemize}

    \subsection{\citetitle{}}
    \begin{itemize}
    \item 
    \item 
    \end{itemize}


\section{pylogenetics and distribution of invasive rats global}

    \subsection{\citetitle{konen_2013}}
    \begin{itemize}
    \item Traces the pattern of invation along shipping routes of in the 1800s using genetic markers
    \item 
    \end{itemize}

    \subsection{\citetitle{gattoalmeida_2022}}
    \begin{itemize}
    \item Dispersal of invasive Norways rats in Brazil via seaports
    \item  "total of 71 rats were genotyped using 11 microsatellite markers. The results revealed a pattern of gene flow contrary to the expected stepping-stone model along the linear transect, with the two furthest apart populations being clustered together. We hypothesize that the observed outcome is explained by natural dispersal along the corridor being lower than human-mediated transport"
    \item 
    \end{itemize}

    \subsection{\citetitle{}}
    \begin{itemize}
    \item 
    \item 
    \end{itemize}

    \subsection{\citetitle{}}
    \begin{itemize}
    \item 
    \item 
    \end{itemize}


\section{Environmental impact of Rats}

    \subsection{\citetitle{sinclair2005did}}
    \begin{itemize}
    \item Impact of invasive mammal removal from Kapiti of invertebrates
    \item Three years after rat eradication we detected a significant decrease in invertebrate catch frequency and diversity, most obvious in the Carabidae and Amphipoda. Site and season accounted for most of the variation in the data. A four‐fold increase in the conspicuousness and condition of some insectivorous birds, and fluctuations between El Niño and La Niña weather patterns may have affected the “recovery” of the island invertebrates. 
    \item 
    \end{itemize}


    \subsection{\citetitle{innes1995large}}
    \begin{itemize}
    \item 
    \item 
    \end{itemize}


    \subsection{\citetitle{mulder2009direct}}
    \begin{itemize}
    \item Seabird presance/nesting on isalnds have a myriad of direct and indirect impcat on island flora and fauna 
    \item 1. Soil chemistry, soil moisture, and light environment
    \item 2. Tree foliar characteristics i.e plant prductivity and nutriant availability
    \item 3. Community structure of woody seedlings i.e rats may reduce native seed dispersing species
    \item 4. Invasion by non-native plants both during invasion and post removal
    \end{itemize}


    \subsection{\citetitle{russell2016fifty}}
    \begin{itemize}
    \item We review the early history of rodent management in New Zealand leading
    up to and including the Big South Cape Island/Taukihepa ship rat invasion, and document the development
    and implementation of rodent eradication technologies on New Zealand islands up to the present day
    \item 
    \end{itemize}

\section{Eradication/Pest control}

\subsection{\citetitle{rost_2004}}
    \begin{itemize}
    \item identification of mutation in the VKORC1 gene confer warafin resitance in human disease
    \item gene vitamin K epoxide reductase complex subunit 1 (VKORC1)
    \end{itemize}

    \subsection{\citetitle{rost_2009}}
    \begin{itemize}
    \item In the present study, we have sequenced the VKORC1 genes of more than 250 rats and mice trapped in anticoagulant-exposed areas from four continents, and identified 18 novel and five published missense mutations, as well as eight neutral sequence variants, in a total of 178 animals. Mutagenesis in VKORC1 cDNA constructs and their recombinant expression revealed that these mutations reduced VKOR activities as compared to the wild-type protein. However, the in vitro enzyme assay used was not suited to convincingly demonstrate the warfarin resistance of all mutant proteins
    \item 
    \end{itemize}

    \subsection{\citetitle{}}
    \begin{itemize}
    \item 
    \item 
    \end{itemize}

\bib{}
    \end{document}
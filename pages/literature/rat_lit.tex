\documentclass[document.tex]{subfiles}
\begin{document}


\section{ }
\subsection{}
\begin{itemize}
\item 
\item 
\end{itemize}

\subsection{PerSVade :\citetitle{wilson2007estimating}}
\begin{itemize}
\item 
\item 
\end{itemize}

\subsection{}
\begin{itemize}
\item 
\item 
\end{itemize}


\section{pylogenetics and distribution on invasive rats}

    \subsection{ :\citetitle{wilson2007estimating}}
    \begin{itemize}
    \item 
    \item The presence of ship rats significantly increased with the presence of a Wharf or if an island was Inhabited
    \item Norway rat presence was negatively correlated with relative seabird species richness, and positively correlated with relative exotic land bird species richness
    \item As for Norway rats, there appears to be a negative interaction between rodent species that affects the distribution of ship rats. However, in this case the relationship is not as significant as it was for Norway rats. This may be because ship rats are the superior competitor in New Zealand (Atkinson, 1986; Yom-Tov et al., 1999; Innes, 2001).
    \end{itemize}


\section{Environmental impact of Rats}

    \subsection{\citetitle{sinclair2005did}}
    \begin{itemize}
    \item Impact of invasive mammal removal from Kapiti of invertebrates
    \item Three years after rat eradication we detected a significant decrease in invertebrate catch frequency and diversity, most obvious in the Carabidae and Amphipoda. Site and season accounted for most of the variation in the data. A four‐fold increase in the conspicuousness and condition of some insectivorous birds, and fluctuations between El Niño and La Niña weather patterns may have affected the “recovery” of the island invertebrates. 
    \item 
    \end{itemize}


    \subsection{\citetitle{innes1995large}}
    \begin{itemize}
    \item 
    \item 
    \end{itemize}


    \subsection{\citetitle{mulder2009direct}}
    \begin{itemize}
    \item Seabird presance/nesting on isalnds have a myriad of direct and indirect impcat on island flora and fauna 
    \item 1. Soil chemistry, soil moisture, and light environment
    \item 2. Tree foliar characteristics i.e plant prductivity and nutriant availability
    \item 3. Community structure of woody seedlings i.e rats may reduce native seed dispersing species
    \item 4. Invasion by non-native plants both during invasion and post removal
    \end{itemize}


    \subsection{\citetitle{russell2016fifty}}
    \begin{itemize}
    \item We review the early history of rodent management in New Zealand leading
    up to and including the Big South Cape Island/Taukihepa ship rat invasion, and document the development
    and implementation of rodent eradication technologies on New Zealand islands up to the present day
    \item 
    \end{itemize}

\bib{}
    \end{document}